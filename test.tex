\hypertarget{navbar-top}{}
\href{s04-00-introduction-to-law-and-legal-.html}{\includegraphics{shared/images/batch-left.png}}
\href{s04-00-introduction-to-law-and-legal-.html}{Previous Section}

\href{index.html}{\includegraphics{shared/images/batch-up.png}}
\href{index.html}{Table of Contents}

\href{s04-02-schools-of-legal-thought.html}{Next Section}
\href{s04-02-schools-of-legal-thought.html}{\includegraphics{shared/images/batch-right.png}}

\hypertarget{book-content}{}
\begin{english}

\hypertarget{mayer_1.0-ch01_s01}{}
{1.1} What Is Law?

\emph{Law} is a word that means different things at different times.
\emph{Black's Law Dictionary} says that law is ``a body of rules of
action or conduct prescribed by controlling authority, and having
binding legal force. That which must be obeyed and followed by citizens
subject to sanctions or legal consequence is a law.''{Black's Law
Dictionary, 6th ed., s.v. ``law.''}

\hypertarget{mayer_1.0-ch01_s01_s01}{}
\subsection{Functions of the Law}\label{functions-of-the-law}

In a nation, the law can serve to (1) keep the peace, (2) maintain the
status quo, (3) preserve individual rights, (4) protect minorities
against majorities, (5) promote social justice, and (6) provide for
orderly social change. Some legal systems serve these purposes better
than others. Although a nation ruled by an authoritarian government may
keep the peace and maintain the status quo, it may also oppress
minorities or political opponents (e.g., Burma, Zimbabwe, or Iraq under
Saddam Hussein). Under colonialism, European nations often imposed peace
in countries whose borders were somewhat arbitrarily created by those
same European nations. Over several centuries prior to the twentieth
century, empires were built by Spain, Portugal, Britain, Holland,
France, Germany, Belgium, and Italy.

With regard to the functions of the law, the empire may have kept the
peace---largely with force---but it changed the status quo and seldom
promoted the native peoples' rights or social justice within the
colonized nation.{Legal debate often centers on the clash between
different legal ideals, such as security versus liberty, liberty versus
equality, and so on. Balancing these ideals is fundamental to the policy
behind the laws we study.}

In nations that were former colonies of European nations, various ethnic
and tribal factions have frequently made it difficult for a single,
united government to rule effectively. In Rwanda, for example, power
struggles between Hutus and Tutsis resulted in genocide of the Tutsi
minority. (Genocide is the deliberate and systematic killing or
displacement of one group of people by another group. In 1948, the
international community formally condemned the crime of genocide.) In
nations of the former Soviet Union, the withdrawal of a central power
created power vacuums that were exploited by ethnic leaders. When
Yugoslavia broke up, the different ethnic groups---Croats, Bosnians, and
Serbians---fought bitterly for home turf rather than share power. In
Iraq and Afghanistan, the effective blending of different groups of
families, tribes, sects, and ethnic groups into a national governing
body that shares power remains to be seen.

\hypertarget{mayer_1.0-ch01_s01_s02}{}
\subsection{Law and Politics}\label{law-and-politics}

In the United States, legislators, judges, administrative agencies,
governors, and presidents make law, with substantial input from
corporations, lobbyists, and a diverse group of nongovernment
organizations (NGOs) such as the American Petroleum Institute, the
Sierra Club, and the National Rifle Association. In the fifty states,
judges are often appointed by governors or elected by the people. The
process of electing state judges has become more and more politicized in
the past fifteen years, with growing campaign contributions from those
who would seek to seat judges with similar political leanings.

In the federal system, judges are appointed by an elected official (the
president) and confirmed by other elected officials (the Senate). If the
president is from one party and the other party holds a majority of
Senate seats, political conflicts may come up during the judges'
confirmation processes. Such a division has been fairly frequent over
the past fifty years.

In most nation-states{The basic entities that comprise the international
legal system. \emph{Countries}, \emph{states}, and \emph{nations} are
all roughly synonymous. \emph{State} can also be used to designate the
basic units of federally united states, such as in the United States of
America, which is a nation-state.} (as countries are called in
international law), knowing who has power to make and enforce the laws
is a matter of knowing who has political power; in many places, the
people or groups that have military power can also command political
power to make and enforce the laws. Revolutions are difficult and
contentious, but each year there are revolts against existing
political-legal authority; an aspiration for democratic rule, or greater
``rights'' for citizens, is a recurring theme in politics and law.

\hypertarget{mayer_1.0-ch01_s01_s02_n01}{}
\subsubsection{Key Takeaway}\label{key-takeaway}

Law is the result of political action, and the political landscape is
vastly different from nation to nation. Unstable or authoritarian
governments often fail to serve the principal functions of law.

\hypertarget{mayer_1.0-ch01_s01_s02_n02}{}
\subsubsection{Exercises}\label{exercises}

\begin{enumerate}
\tightlist
\item
  Is there a sense in which non-governmental entities, such as a church
  or social group, exercise ``law''?
\item
  Law at its highest levels, such as deciding who sits on the Supreme
  Court, is deeply political. What are the pros and cons of this? Can
  you think of a less political way to appoint members of the Supreme
  Court?
\end{enumerate}

\end{english}

\hypertarget{navbar-bottom}{}
\href{s04-00-introduction-to-law-and-legal-.html}{\includegraphics{shared/images/batch-left.png}}
\href{s04-00-introduction-to-law-and-legal-.html}{Previous Section}

\href{index.html}{\includegraphics{shared/images/batch-up.png}}
\href{index.html}{Table of Contents}

\href{s04-02-schools-of-legal-thought.html}{Next Section}
\href{s04-02-schools-of-legal-thought.html}{\includegraphics{shared/images/batch-right.png}}
